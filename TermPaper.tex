\documentclass[twoside,12pt, leqno]{amsart}
\usepackage{amsmath,amscd,amsthm,amssymb,amsxtra,latexsym,epsfig,epic,graphics,mathtools}
\usepackage[matrix,arrow,curve]{xy}
\usepackage{graphicx}
\usepackage{diagrams}
\usepackage{tikz,color}  %TikZ
\usepackage{tikz-cd}
\usepackage{quiver}
%\usepackage{amsrefs}
%%%%%%%%%%%%%%%%%%%%%%%%%%%%%%%%%%%%%%%%%
%\textwidth16cm
%\textheig\codim20cm
%\topmargin-2cm
\oddsidemargin.8cm
\evensidemargin1cm

%%%%%Definitions
\input{Termpreamble.tex}
\def\e{{\epsilon}}
\def\TU{{\bf U}}
\def\AA{{\mathbb A}}
\def\BB{{\mathbb B}}
\def\bB{{\mathbb B}}
\def\PP{{\mathbb P}}
\def\P{{\mathbb P}}
\def\QQ{{\mathbb Q}}
\def\FF{{\mathbb F}}
\def\facet{{\bf facet}}
\def\image{{\rm image}}
\def\cE{{\cal E}}
\def\cF{{\cal F}}
\def\cG{{\cal G}}
\def\cH{{\cal H}}
\def\cHom{{{\cal H}om}}
\def\fix#1{{\bf ***Fix:} #1 {\bf ***}}
\def\david#1{{\bf *** David:} #1 {\bf ***}}
\DeclareMathOperator{\rH}{{\rm H}}
\def\fC{{\mathfrak C}}
\def\Tr{{\rm Tr}}
\def\bC{{\mathbb C}}
\def\Gr{{\rm Gr}}
\def\CI{{\mathcal I}}
\def\CH{{\mathcal H}}
%\def\CCH{{\mathcal {CNT}}}
\def\CCH{{\mathcal {HC}}}
\def\rH{{\rm H}}

\def\soc{{\rm soc\,}}
\def\jacobian{{\rm Jac}}
\def\Rbar{{\overline R}}
\def\Ibar{{\overline I}}
\def\mm{{\frak m}}
\def\RR{{\mathcal R}}
\def\Trace{{\rm Tr}}

\def\CO{{\mathcal O}}
\def\CT{{\mathcal T}}
\def\CHom{{\mathcal Hom}}
\def\Spec{{{\rm Spec}\,}}
\def\cone{{{\rm cone}\,}}

\def\tR{{\tilde R}}
\def\tI{{\tilde I}}
\def\tJ{{\tilde J}}
\def\tK{{\tilde K}}
\def\tH{{\tilde H}}
\def\tF{{\tilde F}}

\newarrow{Iso} -----

\def\Abar{{\overline A}}
\def\Rbar{{\overline R}}
\def\Ibar{{\overline I}}
\def\Jbar{{\overline J}}
\def\Kbar{{\overline K}}
\def\abar{{\overline \alpha}}
\def\bbar{{\overline \beta}}
\def\m{{\frak m}}
\def\Rbar{{\overline R}}

\def\gr{{\rm gr}}

\def\lbracket{{[\kern-1.5pt[}}
\def\rbracket{{]\kern-1.5pt]}}

\def\seq#1#2{{#1_{1},\dots,#1_{#2}}}
\def\ff#1{{f_{1},\dots, f_{#1}}}

\makeatletter
\def\Ddots{\mathinner{\mkern1mu\raise\p@
\vbox{\kern7\p@\hbox{.}}\mkern2mu
\raise4\p@\hbox{.}\mkern2mu\raise7\p@\hbox{.}\mkern1mu}}
\makeatother


%%%%%%%%%%%%%%%%%%Silvio's macros for the diagrams
\usepackage{times}
\newdimen\x \x=12pt

%\usepackage{mat\codimime}
\usepackage{color}

%\usepackage{color}
%\usepackage[usenames,dvipsnames,svgnames,table]{xcolor}

\usepackage[breaklinks,bookmarksopen,bookmarksnumbered,urlcolor=blue]{hyperref}
\hypersetup{colorlinks=true,backref=true,citecolor=blue}

%\pagestyle{myheadings}
%\date{April 2013-December 2015}
\author{Daniel Rostamloo}

\title{The \texttt{KoszulSummands} Package for Macaulay2}
\begin{document}

\begin{abstract}
	This note illustrates the \texttt{KoszulSummands} package for Macaulay2 and makes conjectures about the persistence of socle summands for resolutions of the residue class field of a local rings.
\end{abstract}

\maketitle

\section*{Introduction}

The following work is a result of a yearlong undergraduate honors thesis completed by the author at the University of California, Berkeley under the supervision of David Eisenbud and motivated by the recent work \cite{DE23} of Dao and Eisenbud showing the existence of linear summands in the cycles of resolutions of the residue class field of certain local rings. 

Recall that the \textbf{socle} of a module is defined to be the sum of its simple submodules. If $M$ is a module over a local ring $(R, \mathfrak{m})$, we easily identify that $\socle(M) = (0 : \mathfrak{m})$. For $I \subset R$ an ideal, we can form the quotient $R / I$ and the Koszul complex $K$ of the maximal ideal of $R / I$ and ask when, if ever, there are elements from $\socle(K_i)$ which generate summands for the $i$-th cycles in the resolution of the residue field. Recent ongoing work of Dao and Eisenbud has shown the persistence of such socle summands for these resolutions over Burch and Golod rings, some key concepts for which we will briefly discuss. The aim of the present work and the \texttt{KoszulSummands} package for Macaulay2 is to aid in making conjectures extending these results to arbitrary local rings. The preliminary definitions and results below are treated in detail in \cite{DKT20} and \cite{DE23}. 

\section{Motivation: Burch Ideals and Socle Summands}

Throughout, we will assume that all rings are commutative with unity and all modules are finitely generated.

\begin{definition}
Let $(S, \mathfrak{m}, k)$ be a local ring. We define a \textbf{Burch ideal} as an ideal $I$ with $\mathfrak{m} I \neq \mathfrak{m}(I :_S \mathfrak{m})$. From this we see that any such Burch ideal has $\depth S / I = 0$. Note that there is a more general definition extending the Burch class to rings of nonzero depth; this is developed in detail in \cite{DE23}.

%We define the ideal $\bm_S(I) \coloneqq I \mathfrak{m} : (I : \mathfrak{m})$, and the \textbf{Burch index} $\burch(S)$ to be the dimension of the vector space 
%\[
%\frac{\mathfrak{m}}{\bm_S(I)}
%.\]
%We see that an ideal being Burch is equivalent to $\burch_S(I) \geqslant 1$.
\end{definition}

\begin{theorem}
	Let $(R, \mathfrak{m}, k)$ be a local ring that is not a field. Then $R$ is a Burch ring of depth zero if and only if $k$ is isomorphic to a direct summand of its second syzygy $\syz_R^2 (k)$ and therefore in every $\syz_R^i (k)$ for $i \geqslant 2$.
\end{theorem}

Already this theorem from \cite{DKT20} and other ongoing work of Dao and Eisenbud shows the interesting behavior that arises when seeking summands of syzygy submodules generated by elements from the socle of the ambient Koszul module. We call these elements \textbf{socle summands}. In some recent work in preparation, Dao and Eisenbud make the following refinement which shows the persistence of such socle summands in the cycles of the Koszul complex of the maximal ideal:

\begin{theorem}
	Let $(S, \mathfrak{m}, k)$ be a regular local ring and $(R = S / I, \mathfrak{m}, k)$ its quotient ring with $I \subset \mathfrak{m}^3$. Let $K = \bigwedge_n R^n$ be the Koszul complex of the maximal ideal of $R$. Let $C_i \subset K_i$ be the modules of cycles. If $C_i$ has a socle summand and $i < n$, then $C_{i+1}$ also has a socle summand.
\end{theorem}

Combined with the previous theorem, this tells us that socle summands of Burch ideals persist from the second syzygies until the end of the associated Koszul complex. 

We now turn to the method of adjoining variables to kill cycles: starting from homological degree 2, we can create a chain complex containing the Koszul complex as a summand by formally adjoining variables whose images generate the cycles in degree 1, making the new complex exact there. After one step of this process, our complex looks like

% https://q.uiver.app/?q=WzAsMTQsWzAsMCwiMCJdLFsxLDAsIktfMCJdLFsyLDAsIktfMSJdLFszLDAsIktfMiJdLFs0LDAsIktfMyJdLFszLDEsIlxcb3RpbWVzIl0sWzMsMiwiQ18xIl0sWzQsMiwiQ18xIFxcb3RpbWVzIEtfMSJdLFs0LDEsIlxcb3RpbWVzIl0sWzUsMiwiQ18xIFxcb3RpbWVzIEtfMiJdLFs1LDAsIlxcbGRvdHMiXSxbNSwxLCJcXG90aW1lcyJdLFs1LDMsIlxcb3RpbWVzIl0sWzUsNCwiQ18xXjIiXSxbMSwwXSxbMiwxXSxbMywyXSxbNCwzXSxbNiwyXSxbNyw2XSxbMTAsNF0sWzksN10sWzksNF0sWzcsM10sWzEzLDRdXQ==
\[\begin{tikzcd}
	0 & {K_0} & {K_1} & {K_2} & {K_3} & \ldots \\
	&&& \oplus & \oplus & \oplus \\
	&&& {C_1} & {C_1 \otimes K_1} & {C_1 \otimes K_2} \\
	&&&&& \oplus \\
	&&&&& {C_1^{\otimes 2}}
	\arrow[from=1-2, to=1-1]
	\arrow[from=1-3, to=1-2]
	\arrow[from=1-4, to=1-3]
	\arrow[from=1-5, to=1-4]
	\arrow[from=3-4, to=1-3]
	\arrow[from=3-5, to=3-4]
	\arrow[from=1-6, to=1-5]
	\arrow[from=3-6, to=3-5]
	\arrow[from=3-6, to=1-5]
	\arrow[from=3-5, to=1-4]
	\arrow[from=5-6, to=1-5]
\end{tikzcd}\]

We leave the details of this construction to \cite{GL69}, where they are treated in full detail. Using the DGAlgebras and SocleSummands packages for Macaulay2, we ask the following (imprecise) question: \textit{for $I$ an ideal of a local ring $R$, do the socle summands of the minimal projective resolution of the residue class field obtained in this way persist as they do for the original Koszul complex?} 

\section{The \texttt{KoszulSummands} Package}

Before we discuss the methods of the \texttt{KoszulSummands} package, we show once and for all the packages which we will be loading:

\begin{footnotesize}
\begin{verbatim}
i1 : loadedPackages

o1 = {Quasidegrees, MonomialOrbits, Truncations, 
KoszulSummands, SocleSummands, DGAlgebras, IntegralClosure, 
MinimalPrimes, Points, LexIdeals, Isomorphism, OnlineLookup, 
SimpleDoc, InverseSystems, ConwayPolynomials, ReesAlgebra, 
TangentCone, Classic, Saturation, Elimination, 
PrimaryDecomposition, LLLBases, Core}

o1 : List
\end{verbatim}
\end{footnotesize}

In this section, we describe our main routine and some important methods for computer experiments. We briefly discuss how we produce suitable ideals on which to test socle summand persistence. We will focus our tests around Burch ideals, since we know from \cite{DE23} that such ideals exhibit socle summands at many steps of the resolution of the residue class field. Given a polynomial ring, we can use the \texttt{MonomialOrbits} package and a suitable starting ideal $I$ which the \texttt{orbitRepresentatives} method uses to generate many ideals by choosing representatives of the orbits from the permutation action on the variables. The \texttt{isBurch} method from \texttt{SocleSummands} then isolates those among the orbit representatives which are Burch ideals.

\begin{example}
	We use the \texttt{DGAlgebras} package to adjoin variables in homological degree 2 with new boundary maps onto the cycles of homological degree 1, making the chain complex exact there.
\end{example}

\begin{footnotesize}
\begin{verbatim}
i2 : kk = ZZ/(101);

i3 : S = kk[a..d];

i4 : B = ideal"a5,a2b,ab2,b5,abc,c5,abd,d5"

             5   2      2   5          5          5
o4 = ideal (a , a b, a*b , b , a*b*c, c , a*b*d, d )

o4 : Ideal of S

i5 : KK = killCycleComplexes(S, B, 2);
\end{verbatim}
\end{footnotesize}

The method \texttt{killCycleComplexes} is based on the \texttt{killCycles} method from the \texttt{DGAlgebras} package. Given the polynomial ring $S$, an ideal $B$, and the integer 2, it returns a list consisting of the Koszul complex of the maximal ideal of $S / B$ and also the chain complex which is made exact up to homological degree 1 by killing cycles. More generally, we can provide the integer $i$ as the final argument to the method so that the method returns a list of $i$ chain complexes such that the first is the Koszul complex, the second is exact up to degree 1, and so on until the last one which is exact up to homological degree $i-1$.

\begin{footnotesize}
\begin{verbatim}
i6 : cycleSummands KK_0

o6 = {0, 6, 7, 6, 3}

o6 : List

i7 : cycleSummands KK_1

o7 = {0, 6, 9, 51, 72, 243}

o7 : List
\end{verbatim}
\end{footnotesize}

The method \texttt{cycleSummands} takes a given chain complex and uses the \texttt{socleSummands} method from the \texttt{SocleSummands} package to check for the number of socle summands occuring for the submodule of cycles in each homological degree. Here, we compute the number of socle summands in the cycles of the original Koszul complex and the complex resulting from killing cycles in homological degree 1. Note that the list begins with 0; one of way of seeing this is a result of ongoing work of Dao and Eisenbud which states that a socle summand is a nontrivial element of the Koszul homology.

\begin{footnotesize}
\begin{verbatim}
i8 : proj = project(KK_0, KK_1);

i9 : prev = previous(KK_0, KK_1);

i10 : orig = original(KK_0, KK_1);

i11 : ig = ignore(KK_0, KK_1);
\end{verbatim}
\end{footnotesize}

Here, we distinguish components of the boundary maps after killing cycles according to their source and target. The \texttt{project()} method composes the boundary maps with a projection to the summand consisting of the original Koszul complex, effectively choosing the submatrix whose rows correspond to generators of the Koszul complex. The \texttt{previous()} group is like the \texttt{project()} group, but it includes the projection of \textit{all} boundary maps in the killed cycles complex. The \texttt{original()} method collects the submatrices of the boundary maps whose rows and columns correspond to generators of the Koszul complex. The \texttt{ignore()} method chooses the submatrix without any rows or columns corresponding to the generators of the Koszul complex.

\begin{footnotesize}
\begin{verbatim}
i12 : mm = ideal gens ring KK_0;

                                   S
o12 : Ideal of -----------------------------------------
                 5   2      2   5          5          5
               (a , a b, a*b , b , a*b*c, c , a*b*d, d )

i13 : apply(length KK_1 - 1, i ->  numcols compress (gens (socle image syz orig_i) % (mm*image prev_(i+1))))

o13 = {0, 6, 7, 4}

o13 : List

i14 : apply(length KK_1 - 1, i -> socleSummands(image (syz(orig_i) % proj_(i+1))))

o14 = {0, 0, 0, 0}

o14 : List

i15 : apply(length KK_1, i -> socleSummands(image syz proj_(i)))

o15 = {0, 0, 2, 4, 2}

o15 : List

i16 : apply(length KK_1, i -> socleSummands(image syz ig_(i)))

o16 = {0, 0, 0, 48, 71}

o16 : List
\end{verbatim}
\end{footnotesize}

The first output from this block check whether there are any socle summands in the top-level Koszul complex which are not contained in the product of the maximal ideal and the image of the previous boundary map. The second output from this block tells us that there are no socle summands in the ``top-level'' Koszul complex outside of the image of the projected boundary maps. However, the third output from this block tells us that socle summands in the kernel of the projected boundary maps recover all of those socle summands up to homological degree 2. The final output gives many socle summands in homological degrees 4 and 5.

Remarkably, the kernel of the projected boundary maps accurately coincides with the difference in socle summands after killing cycles, especially in homological degree 3. We can conjecture that this is a generally accurate source for the additional socle summands.

Dao and Eisenbud conjecture that the persistence of socle summands seen in the Koszul complex of the maximal ideal is also true for the resolution of the residue class field:

\begin{conjecture}[Dao, Eisenbud]
	Let $(S, \mathfrak{m}, k)$ be a regular local ring of dimension $n$. Let $I$ be a Burch ideal of $S$ and $R \coloneqq S / I$ be the resulting quotient ring. If $K$ is the Koszul complex of the maximal ideal of $R$, then let $\widetilde{K}$ denote the augmented acyclic Koszul algebra. If we denote the cycles of the restriction of the $i$-th boundary map to the original Koszul complex by $C_i$, then each $C_i$ for $1 \leqslant i \leqslant n$ has a socle summand. 
\end{conjecture}
Let $(S, mm, k)$ be a regular local ring of dimension $n$, and $I\subset mm^3$ an ideal. Set R = S/I.
Let $K$ be the Koszul complex of the maximal ideal in $R$ , and let $\widetilde K$ be the 
minimal $R$-free resolution of $k$. 

Persistence conjecture (Dao, Eisenbud): if $Cycles_i(\widetilde K)$ has a socle summand, then so does 
$Cycles_{i+1}(\widetilde K)$ unless $i=n$, in which case $Cycles_{i+2}(\widetilde K)$ also has a socle summand.

As a step toward proving this conjecture, we make:

Persistence conjecture (Dao, Eisenbud): if $Cycles_i(K)$ has a socle summand and $i<n$ then  
$Cycles_i(\widetilde K)$ has a socle summand  contained in $K_i$.

\let\thefootnote\relax\footnote{
\noindent A\S Subject Classification:\\
Primary: 13C40, 13H10, 14M06, 14M10;
Secondary: 13D02 , 13N05, 14B12, 14M12.\smallbreak
The author is grateful to the
National Science Foundation for partial support.<}
\bibliographystyle{alpha}
\begin{thebibliography}{ABC99}

	\bibitem{DE23} \textbf{Dao, Hailong; Eisenbud, David}. Burch index, summands of syzygies and linearity in resolutions. \textit{Bull. Iranian Math. Soc.} \textbf{49} (2023), no. 2, Paper No. 10, 10 pp. MR4549775

	\bibitem{DKT20} \textbf{Dao, Hailong; Kobayashi, Toshinori; Takahashi, Ryo}. Burch ideals and Burch rings. \textit{Algebra Number Theory} \textbf{14} (2020), no. 8, 2121--2150. MR4172703

	\bibitem{GL69} \textbf{Gulliksen, Tor H.; Levin, Gerson}. Homology of local rings. Queen's Papers in Pure and Applied Mathematics, No. 20 \textit{Queen's University, Kingston, Ont}. 1969 {\rm x}+192 pp. MR0262227

	\bibitem{M2} \textbf{Grayson, Daniel R. and Stillman, Michael E}. Macaulay2, a software system for research in algebraic geometry. Available at \url{http://www.math.uiuc.edu/Macaulay2/}.

	\bibitem{Eis95} \textbf{D. Eisenbud}, Commutative algebra, Graduate Texts in Mathematics, 150, \textit{Springer, New York}, 1995. MR1322960

\end{thebibliography}

\bigskip

\vbox{\noindent Author Addresses:\par
\smallskip
\noindent{David Eisenbud}\par
\noindent{Mathematical Sciences Research Institute,
Berkeley, CA 94720, USA}\par
\noindent{de@msri.org}\par
}

\end{document}


