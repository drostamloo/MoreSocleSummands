\documentclass[twoside,12pt, leqno]{amsart}
\usepackage{amsmath,amscd,amsthm,amssymb,amsxtra,latexsym,epsfig,epic,graphics,mathtools}
\usepackage[matrix,arrow,curve]{xy}
\usepackage{graphicx}
\usepackage{diagrams}
\usepackage{tikz,color}  %TikZ
\usepackage{tikz-cd}
\usepackage{quiver}
%\usepackage{amsrefs}
%%%%%%%%%%%%%%%%%%%%%%%%%%%%%%%%%%%%%%%%%
%\textwidth16cm
%\textheig\codim20cm
%\topmargin-2cm
\oddsidemargin.8cm
\evensidemargin1cm

%%%%%Definitions
%\special{papersize=210mm,297mm}
%
\voffset1cm
%%%%%%%%%%%%%%%%%%%%%%%%%%%%%%%%%%%%%%%%%%%%%%%%
%\textwidth14cm
%\textheight23cm
%\oddsidemargin0.8cm
%\evensidemargin1cm
%%%%%%%%%%%%%%%%%%%%%%%%%%%%%%%%%%%%%%%%%%%%%%%
%\usepackage{fullpage,amsmath,amscd,amsthm,amssymb,amsxtra,latexsym}
%\usepackage{epsfig,epic,eepic,graphics,rotating}
%\usepackage{showkeys}
%\sloppy
%\setlength{\parindent}{0pt}
%\setlength{\parskip}{5pt plus  2pt minus 1pt}
%\topmargin-2cm
%\input matheb.mac
%\newcommand{\somespace}{\hfill{}\\ \vspace{-0.25cm}}
%\pagestyle{headings}
%\setcounter{secnumdepth}{3}
%\setcounter{tocdepth}{3}

%\documentclass{book}
%\usepackage{amsmath,amscd,amsthm,amssymb,amsxtra,latexsym,epsfig,epic,eepic,graphics}

%\usepackage{amsmath,amscd,amsthm,amssymb,amsxtra,latexsym,epsfig,epic,graphics}

%\usepackage[matrix,arrow,curve]{xy}


%\smartqed

\def\antiddot{\mathinner{\mkern1mu\raise1pt\vbox{\kern7pt\hbox{.}}\mkern2mu
        \raise4pt\hbox{.}\mkern2mu\raise7pt\hbox{.}\mkern1mu}}

%%%%%%%%%%%%%%%%%%%%%%%%%%%%
%%%The black board font
%%%%%%%%%%%%%%%%%%%%%%%%%%%
%\newcommand{\bK}{{\bf K}}
\newcommand{\bK}{{\Bbbk}}
%\newcommand{\bK}{{\mathbb k}}
\newcommand{\AAA}{{\mathbb A}}
\newcommand{\BB}{{\mathbb B}}
\newcommand{\CC}{{\mathbb C}}
\newcommand{\DD}{{\mathbb D}}
\newcommand{\EE}{{\mathbb E}}
\newcommand{\FF}{{\mathbb F}}
\newcommand{\GG}{{\mathbb G}}
%\newcommand{\HH}{{\mathbb H}}
\newcommand{\II}{{\mathbb I}}
\newcommand{\JJ}{{\mathbb J}}
\newcommand{\KK}{{\mathbb K}}
%\newcommand{\LL}{{\mathbb L}}
\newcommand{\MM}{{\mathbb M}}
\newcommand{\NN}{{\mathbb N}}
%\newcommand{{\mathbb O}}
\newcommand{\PP}{{\mathbb P}}
\newcommand{\QQ}{{\mathbb Q}}
\newcommand{\RR}{{\mathbb R}}
%\newcommand{\S}{{\mathbb S}}
\newcommand{\TT}{{\mathbb T}}
\newcommand{\UU}{{\mathbb U}}
\newcommand{\VV}{{\mathbb V}}
\newcommand{\WW}{{\mathbb W}}
\newcommand{\XX}{{\mathbb X}}
\newcommand{\YY}{{\mathbb Y}}
\newcommand{\ZZ}{{\mathbb Z}}
\newcommand{\LF}{{\rm\bf L}}
\newcommand{\RF}{{\rm\bf R}}
\newcommand{\TF}{{\rm\bf T}}
\newcommand{\UF}{{\rm\bf U}}


\newcommand{\LL}{{\mathbb L}}
\newcommand{\HH}{{\rm{H}}}
\newcommand{\Syz}{{\rm{Syz}\;}}
\newcommand{\SSyz}{{\rm{Syz}}}
\newcommand{\spoly}{{\rm{spoly}}}
\newcommand{\Spe}{{Sp}}
\newcommand{\openC}{{\mathbb C}}
\newcommand{\ms}{{\rm{m}}}
\newcommand{\LS}{{\rm{L}}}
\newcommand{\IS}{{\rm{I}}}
\newcommand{\Loc}{{\rm{Loc}\,}}
\newcommand{\lcm}{{\rm{lcm}}}
\newcommand{\lc}{{\rm{lc}}}
\newcommand{\lm}{{\rm{lm}}}
\newcommand{\con}{{\rm{c}}}
\newcommand{\ext}{{\rm{e}}}
\newcommand{\ec}{{\rm{ec}}}
\newcommand{\ann}{{\rm{ann}}}
\newcommand{\Ext}{{\rm{Ext}}}
\newcommand{\equi}{{\rm{equi}}}
\newcommand{\rad}{{\rm{rad\;}}}
\newcommand{\mult}{{\rm{mult}}}
\newcommand{\V}{{\rm V}}
%\newcommand{\H}{{\rm H}}
\newcommand{\Id}{{\rm I}}
\newcommand{\ini}{{\bf{L}}}
\newcommand{\dottedarrow}{{- - >}}
\newcommand{\fr}{{f_1,\ldots,f_r}}
\newcommand{\PR}{{\bK [x_1, \dots , x_n]}}
\newcommand{\id}{{\rm{id}}}
\newcommand{\lex}{{\rm{lex}}}
\newcommand{\drlex}{{\rm{drlex}}}
\newcommand{\coker}{{\rm{coker}\,}}
\newcommand{\QF}{{\rm{Q}}}



%%%%%%%%%%%%%%%%%%%%%%%%%%%%%%
%%%The mathscript for sheaves
%%%%%%%%%%%%%%%%%%%%% %%%%%%%%%
\newcommand{\s}{\mathcal}
\newcommand{\sA}{{\s A}}
\newcommand{\sB}{{\s B}}
\newcommand{\sC}{{\s C}}
\newcommand{\sD}{{\s D}}
\newcommand{\sE}{{\s E}}
\newcommand{\sF}{{\s F}}
\newcommand{\sG}{{\s G}}
\newcommand{\sH}{{\s H}}
\newcommand{\sI}{{\s I}}
\newcommand{\sJ}{{\s J}}
\newcommand{\sK}{{\s K}}
\newcommand{\sL}{{\s L}}
\newcommand{\sM}{{\s M}}
\newcommand{\sN}{{\s N}}
\newcommand{\sO}{{\s O}}
\newcommand{\sP}{{\s P}}
\newcommand{\sQ}{{\s Q}}
\newcommand{\sR}{{\s R}}
\newcommand{\sS}{{\s S}}
\newcommand{\sT}{{\s T}}
\newcommand{\sU}{{\s U}}
\newcommand{\sV}{{\s V}}
\newcommand{\sW}{{\s W}}
\newcommand{\sX}{{\s X}}
\newcommand{\sY}{{\s Y}}
\newcommand{\sZ}{{\s Z}}

\newcommand{\cO}{{\s O}}
\newcommand{\cF}{{\s F}}
\newcommand{\cG}{{\s G}}
\newcommand{\cI}{{\s I}}
\newcommand{\cL}{{\s L}}
\newcommand{\cM}{{\s M}}
\newcommand{\cR}{{\s R}}
\newcommand{\cN}{{\s N}}
\newcommand{\cT}{{\s T}}
\newcommand{\cX}{{\s X}}
\newcommand{\lh}{{\ell }}
\DeclareMathOperator{\Gal}{Gal}
%%%%%%%%%%%%%%%%%%%%%%%%%%%%%%%%
%% Arrows
%%%%%%%%%%%%%%%%%%%%%%%%%%%%%%%
\newcommand{\inj}{\hookrightarrow}
\newcommand{\surj}{\lra}
\newcommand{\lra}{\longrightarrow}
\newcommand{\lla}{\longleftarrow}
%%%%%%%%%%%%%%%%%%%%%%%%%%%%%%%%%%%%
%\newcommand{\C}{\C}
\newcommand{\openP}{\P}
\newcommand{\uf}{{\bf F}}
\newcommand{\uc}{{\bf C}}
\newcommand{\tensor}{\otimes}
\newcommand{\mi}{{\bf m}}
\newcommand{\tX}{\widetilde{X}}
\newcommand{\punkt}{\hspace{-.3ex}\raise.15ex\hbox to1ex{\Huge.}}
\newcommand{\tpunkt}{\hspace{-.3ex}\hbox to1ex{\Huge.}}
%\newcommand{\fix {#1}}{{\bf(( **** fix: #1)) }}
%def\fix {#1}{{\bf(( **** fix: {#1})) }}
\def \fix#1 {{\hfill\break \bf (( #1 ))\hfill\break}}

\newcommand{\later}{{\bf ** add reference later **}}
\newcommand{\locring}{\mathcal O_{A,p}}
\newlength{\br}
\DeclareMathOperator{\Pic}{Pic}
\newlength{\ho}

\DeclareMathOperator{\Ann}{Ann}
\DeclareMathOperator{\Ass}{Ass}
\DeclareMathOperator{\GL}{GL}
\DeclareMathOperator{\Aut}{Aut}
\DeclareMathOperator{\Oo}{O}
\DeclareMathOperator{\Sym}{Sym}
\DeclareMathOperator{\reg}{reg}
\DeclareMathOperator{\Spec}{Spec}
\DeclareMathOperator{\Proj}{Proj}
\DeclareMathOperator{\Hom}{Hom}
\DeclareMathOperator{\sHom}{\sH om}
\DeclareMathOperator{\domain}{dom}
\DeclareMathOperator{\Den}{D}
\DeclareMathOperator{\Homol}{H}
\DeclareMathOperator{\syz}{syz}
\DeclareMathOperator{\ord}{ord}
\DeclareMathOperator{\word}{w\,ord}
\DeclareMathOperator{\supp}{supp}
\DeclareMathOperator{\Ker}{Ker}
\DeclareMathOperator{\im}{im}
%\DeclareMathOperator{\wdeg}{w\,deg}
\DeclareMathOperator{\depth}{depth}
\DeclareMathOperator{\length}{length}
\DeclareMathOperator{\gin}{gin}
\DeclareMathOperator{\Coker}{Coker}
\DeclareMathOperator{\NF}{NF}
\DeclareMathOperator{\pd}{pd}
\DeclareMathOperator{\SL}{SL}
\DeclareMathOperator{\SO}{SO}
\DeclareMathOperator{\Ort}{O}
%\DeclareMathOperator{\Spez}{Sp}
\DeclareMathOperator{\pfaff}{pfaff}
\DeclareMathOperator{\PSL}{PSL}
\DeclareMathOperator{\PGL}{PGL}
\DeclareMathOperator{\Tor}{Tor}
\DeclareMathOperator{\divi}{div}
\DeclareMathOperator{\Div}{Div}
\DeclareMathOperator{\wdim}{wdim}
\DeclareMathOperator{\cdim}{cdim}
%\DeclareMathOperator{\cha}{char}
\DeclareMathOperator{\trdeg}{trdeg}
\DeclareMathOperator{\codim}{codim}
\DeclareMathOperator{\rank}{rank}
\DeclareMathOperator{\kdim}{kdim}
\DeclareMathOperator{\height}{height}
\DeclareMathOperator{\Lie}{Lie}
\DeclareMathOperator{\cha}{char}
\DeclareMathOperator{\GCD}{GCD}
\DeclareMathOperator{\LCM}{LCM}
\DeclareMathOperator{\Spoly}{S}
\DeclareMathOperator{\T}{T}
\DeclareMathOperator{\TC}{TC}
\DeclareMathOperator{\sing}{sing}
\DeclareMathOperator{\spann}{span}
\DeclareMathOperator{\lspan}{span}
\DeclareMathOperator{\don}{D}
\DeclareMathOperator{\diff}{d}
\DeclareMathOperator{\SOS}{D}
\DeclareMathOperator{\bm}{bm}
\DeclareMathOperator{\burch}{burch}
\DeclareMathOperator{\socle}{socle}


%\renewcommand{\labelenumi}{(\arabic{enumi})}
%\newcommand{\Ndash}{\nobreakdash--}% for pages 1\Ndash 9
%\newcommand{\somespace}{\hfill{}\\ \vspace{-0.7cm}}

%%%theosdefinitionen
\newcommand{\gm}{\mathfrak m}
\newcommand{\gn}{\mathfrak n}
\def\gr{{\mathfrak {gr}}}
\newcommand{\integer}{\ZZ}
\newcommand{\proj}{\PP}
\newcommand{\complex}{\CC}
\newcommand{\real}{\mathbb R}
\newcommand{\gp}{\mathfrak p}
\newcommand{\ga}{\mathfrak a}
\newcommand{\gq}{\mathfrak q}
\newcommand{\gP}{\mathfrak P}
\newcommand{\gQ}{\mathfrak Q}

\newcommand{\bx}{\boldsymbol{x}}
\newcommand{\by}{\boldsymbol{y}}
%\newcommand{\openF}{\F}
\newlength{\kr}
%%%%%%%%%%%%%%%BIBLIOGRAPHY

%%% Computer algebra systems
\newcommand{\Mac}{{\texttt {MACAULAY2}}}
\newcommand{\Maca}{{\texttt {MACAULAY2 }}}
\newcommand{\Sing}{{\texttt {SINGULAR}}}
\newcommand{\Sch}{{\texttt {SCHUBERT}}}
%\theoremstyle{plain}
\newtheorem{theorem}{Theorem}[section]
\newtheorem{lemma}[theorem]{Lemma}
\newtheorem{proposition}[theorem]{Proposition}
\newtheorem{corollary}[theorem]{Corollary}
%\newtheorem{satz}{Satz}[section]
\newtheorem{conjecture}[theorem]{Conjecture}
\theoremstyle{definition}
\newtheorem{definition}[theorem]{Definition}
\newtheorem{notation}[theorem]{Notation}
\newtheorem{thmdef}[theorem]{Theorem-Definition}
\newtheorem{remark}[theorem]{Remark}
\newtheorem{remdef}[theorem]{Remark-Definition}
\newtheorem{example}[theorem]{Example}
\newtheorem{exercise}[theorem]{Exercise}
\newtheorem{algorithm}[theorem]{Algorithm}
\newtheorem{sub}[subsubsection]{}

\def\e{{\epsilon}}
\def\TU{{\bf U}}
\def\AA{{\mathbb A}}
\def\BB{{\mathbb B}}
\def\bB{{\mathbb B}}
\def\PP{{\mathbb P}}
\def\P{{\mathbb P}}
\def\QQ{{\mathbb Q}}
\def\FF{{\mathbb F}}
\def\facet{{\bf facet}}
\def\image{{\rm image}}
\def\cE{{\cal E}}
\def\cF{{\cal F}}
\def\cG{{\cal G}}
\def\cH{{\cal H}}
\def\cHom{{{\cal H}om}}
\def\fix#1{{\bf ***Fix:} #1 {\bf ***}}
\def\david#1{{\bf *** David:} #1 {\bf ***}}
\DeclareMathOperator{\rH}{{\rm H}}
\def\fC{{\mathfrak C}}
\def\Tr{{\rm Tr}}
\def\bC{{\mathbb C}}
\def\Gr{{\rm Gr}}
\def\CI{{\mathcal I}}
\def\CH{{\mathcal H}}
%\def\CCH{{\mathcal {CNT}}}
\def\CCH{{\mathcal {HC}}}
\def\rH{{\rm H}}

\def\soc{{\rm soc\,}}
\def\jacobian{{\rm Jac}}
\def\Rbar{{\overline R}}
\def\Ibar{{\overline I}}
\def\mm{{\frak m}}
\def\RR{{\mathcal R}}
\def\Trace{{\rm Tr}}

\def\CO{{\mathcal O}}
\def\CT{{\mathcal T}}
\def\CHom{{\mathcal Hom}}
\def\Spec{{{\rm Spec}\,}}
\def\cone{{{\rm cone}\,}}

\def\tR{{\tilde R}}
\def\tI{{\tilde I}}
\def\tJ{{\tilde J}}
\def\tK{{\tilde K}}
\def\tH{{\tilde H}}
\def\tF{{\tilde F}}

\newarrow{Iso} -----

\def\Abar{{\overline A}}
\def\Rbar{{\overline R}}
\def\Ibar{{\overline I}}
\def\Jbar{{\overline J}}
\def\Kbar{{\overline K}}
\def\abar{{\overline \alpha}}
\def\bbar{{\overline \beta}}
\def\m{{\frak m}}
\def\Rbar{{\overline R}}

\def\gr{{\rm gr}}

\def\lbracket{{[\kern-1.5pt[}}
\def\rbracket{{]\kern-1.5pt]}}

\def\seq#1#2{{#1_{1},\dots,#1_{#2}}}
\def\ff#1{{f_{1},\dots, f_{#1}}}

\makeatletter
\def\Ddots{\mathinner{\mkern1mu\raise\p@
\vbox{\kern7\p@\hbox{.}}\mkern2mu
\raise4\p@\hbox{.}\mkern2mu\raise7\p@\hbox{.}\mkern1mu}}
\makeatother


%%%%%%%%%%%%%%%%%%Silvio's macros for the diagrams
\usepackage{times}
\newdimen\x \x=12pt

%\usepackage{mat\codimime}
\usepackage{color}

%\usepackage{color}
%\usepackage[usenames,dvipsnames,svgnames,table]{xcolor}

\usepackage[breaklinks,bookmarksopen,bookmarksnumbered,urlcolor=blue]{hyperref}
\hypersetup{colorlinks=true,backref=true,citecolor=blue}

%\pagestyle{myheadings}
%\date{April 2013-December 2015}
\author{David Eisenbud and Daniel Rostamloo}

\title{The \texttt{KoszulSummands} Package for Macaulay2}
\begin{document}

\begin{abstract}
	This note illustrates the \texttt{KoszulSummands} package for Macaulay2 and makes a conjecture about the persistence of socle summands for augmented Koszul resolutions of Burch ideals.
\end{abstract}

\maketitle

\section*{Introduction}

The \textbf{socle} of a module is defined to be the sum of its simple submodules. If $M$ is a module over a local ring $(R, \mathfrak{m})$, we easily identify that $\socle(M) = (0 : \mathfrak{m})$. For $I$ a suitable ideal as discussed below, we can form the quotient $R / I$ and the associated Koszul complex $K$ and ask when, if ever, there are elements from $\socle(K_i)$ which generate summands for the $i$-th cycles of the Koszul complex. Recent ongoing work of Dao and Eisenbud has shown interesting linearity behavior in the Koszul complex for Burch rings, some key concepts for which we will briefly discuss. The aim of the present work and the \texttt{KoszulSummands} package for Macaulay2 is to aid in making conjectures extending these results to the augmented acyclic Koszul algebra of such Burch rings. The preliminary definitions and results below are treated in detail in \cite{DKT20} and \cite{DE23}. 

\section{Burch Ideals and Socle Summands}

Throughout, we will assume that all rings are commutative with unity and all modules are finitely generated.

\begin{definition}
Let $(S, \mathfrak{m}, k)$ be a local ring. We define a \textbf{Burch ideal} as an ideal $I$ with $\mathfrak{m} I \neq \mathfrak{m}(I :_S \mathfrak{m})$. From this we see that any such Burch ideal has $\depth S / I = 0$. 

%We define the ideal $\bm_S(I) \coloneqq I \mathfrak{m} : (I : \mathfrak{m})$, and the \textbf{Burch index} $\burch(S)$ to be the dimension of the vector space 
%\[
%\frac{\mathfrak{m}}{\bm_S(I)}
%.\]
%We see that an ideal being Burch is equivalent to $\burch_S(I) \geqslant 1$.
\end{definition}

\begin{theorem}
	Let $(R, \mathfrak{m}, k)$ be a local ring that is not a field. Then $R$ is a Burch ring of depth zero if and only if $k$ is isomorphic to a direct summand of its second syzygy $\syz_R^2 (k)$.
\end{theorem}

Already this theorem from \cite{DKT20} shows the interesting behavior that arises when seeking summands of syzygy submodules generated by elements from the socle of the ambient Koszul module. We call these elements \textbf{socle summands}. In some recent work in preparation, Dao and Eisenbud make the following refinement which shows the persistence of such socle summands in the syzygies of the residue field:

\begin{theorem}
	Let $(S, \mathfrak{m}, k)$ be a regular local ring and $(R = S / I, \mathfrak{m}, k)$ its quotient ring with $I \subset \mathfrak{m}^2$. Let $K = \bigwedge_n R^n$ be the Koszul complex (as a differential graded algebra) of the maximal ideal of $R$. Let $C_i \subset K_i$ be the modules of cycles. If $C_i$ has a socle summand and $i < n$, then $C_{i+1}$ also has a socle summand.
\end{theorem}

Combined with the previous theorem, this tells us that socle summands of Burch ideals persist from the second syzygies until the end of the associated Koszul complex. We now turn to the method of adjoining cycles to make this Koszul complex exact: starting from homological degree 2, we can augment our Koszul complex by formally adjoining variables whose images generate the cycles in degree 1, making the complex exact there. Keeping in mind that our Koszul complex is a differential graded algebra, the resulting complex will have a copy of the original Koszul complex tensored with these cycles as a summand. After one step of this process, our complex looks like

% https://q.uiver.app/?q=WzAsMTQsWzAsMCwiMCJdLFsxLDAsIktfMCJdLFsyLDAsIktfMSJdLFszLDAsIktfMiJdLFs0LDAsIktfMyJdLFszLDEsIlxcb3RpbWVzIl0sWzMsMiwiQ18xIl0sWzQsMiwiQ18xIFxcb3RpbWVzIEtfMSJdLFs0LDEsIlxcb3RpbWVzIl0sWzUsMiwiQ18xIFxcb3RpbWVzIEtfMiJdLFs1LDAsIlxcbGRvdHMiXSxbNSwxLCJcXG90aW1lcyJdLFs1LDMsIlxcb3RpbWVzIl0sWzUsNCwiQ18xXjIiXSxbMSwwXSxbMiwxXSxbMywyXSxbNCwzXSxbNiwyXSxbNyw2XSxbMTAsNF0sWzksN10sWzksNF0sWzcsM10sWzEzLDRdXQ==
\[\begin{tikzcd}
	0 & {K_0} & {K_1} & {K_2} & {K_3} & \ldots \\
	&&& \otimes & \otimes & \otimes \\
	&&& {C_1} & {C_1 \otimes K_1} & {C_1 \otimes K_2} \\
	&&&&& \otimes \\
	&&&&& {C_1^2}
	\arrow[from=1-2, to=1-1]
	\arrow[from=1-3, to=1-2]
	\arrow[from=1-4, to=1-3]
	\arrow[from=1-5, to=1-4]
	\arrow[from=3-4, to=1-3]
	\arrow[from=3-5, to=3-4]
	\arrow[from=1-6, to=1-5]
	\arrow[from=3-6, to=3-5]
	\arrow[from=3-6, to=1-5]
	\arrow[from=3-5, to=1-4]
	\arrow[from=5-6, to=1-5]
\end{tikzcd}\]

We leave the details of this construction to \cite{GL69}, where they are treated in full detail. Using the DGAlgebras and SocleSummands packages for Macaulay2, we ask the following question: \textit{for $I$ a Burch ideal, do the socle summands of the augmented acyclic Koszul complex persist when we restrict the target of the boundary maps to the original Koszul complex?} 

\begin{example}
Before we discuss the methods of the \texttt{KoszulSummands} package which we used to formulate a positive conjecture for this question, we exhibit some Macaulay2 code here to familiarize the reader with the general routine of our experiments. The code below shows an example of socle summand persistence for a particular Burch ideal of a polynomial ring.
\end{example}
\begin{footnotesize}
\begin{verbatim}
i1 : kk = ZZ/(101);
i2 : S = kk[a..d];
i3 : I = ideal"a5, b5, c5, d5";
o3 : Ideal of S
i4 : L = orbitRepresentatives(S, I, (3,3,3,3)); #L
o5 = 248
i6 : B = select(L, l -> isBurch(l)); #B
o7 = 7
i8 : B_0
                     3   2    5   2    5   2    5
o8 = monomialIdeal (a , a b, b , a c, c , a d, d )
o8 : MonomialIdeal of S
i9 : KK = killCycleComplexes(S, B_0, 4);
i10 : cycleSummands KK_0
o10 = {4, 6, 4, 2}
o10 : List
\end{verbatim}
\end{footnotesize}

Note that the method \texttt{cycleSummands} does not report the existence of socle summands for homological degrees lower than 2, since these are not interesting in the Burch case.

\section{The \texttt{KoszulSummands} Package}

In this section, we describe our main routine and some important methods for computer experiments. But we first require a way to generate Burch ideals for carrying such tests out; the example above illustrates our approach. Given a polynomial ring, we can use the \texttt{MonomialOrbits} package and a suitable starting ideal $I$ which the \texttt{orbitRepresentatives} method uses to generate many ideals by choosing representatives of the orbits from the permutation action on the variables. The \texttt{isBurch} method from \texttt{SocleSummands} then isolates those among the orbit representatives which are Burch ideals.

We use the \texttt{restrictTarget} method to restrict the boundary maps of the augmented acyclic algebra to the original Koszul summand by comparing the difference in module ranks from the original complex to the acyclic one. We can then use \texttt{cycleSummands} to check if the cycles of the restricted maps exhibit persisting socle summand behavior.

\begin{example}
The example below illustrates the routine of restricting the boundary maps of the augmented Koszul algebra and checking for socle summands.
\end{example}
\begin{footnotesize}
\begin{verbatim}
i1 : kk = ZZ/(101);
i2 : S = kk[a..d];
i3 : I = ideal"a5, b5, c5, d5";
o3 : Ideal of S
i4 : L = orbitRepresentatives(S, I, (3,3,3,3)); #L
o5 = 248
i6 : B = select(L, l -> isBurch(l)); #B
o7 = 7
i8 : B_0
                     3   2    5   2    5   2    5
o8 = monomialIdeal (a , a b, b , a c, c , a d, d )
o8 : MonomialIdeal of S
i9 : KK = killCycleComplexes(S, B_0, 4);
i10 : cycleSummands KK_0
o10 = {4, 6, 4, 2}
o10 : List
i11 : restricted = restrictTarget(KK_0, KK_3);
i12 : apply(restricted, i -> socleSummands(image syz i, Verbose => false))
o12 = {4, 9, 10, 4}
o12 : List
\end{verbatim}
\end{footnotesize}

We close with the following conjecture, made with the help of computations like these.

\begin{conjecture}
	Let $(S, \mathfrak{m}, k)$ be a regular local ring of dimension $n$. Let $I$ be a Burch ideal of $S$ and $R \coloneqq S / I$ be the resulting quotient ring. If $K$ is the Koszul complex of the maximal ideal of $R$, then let $\widetilde{K}$ denote the augmented acyclic Koszul algebra. If we denote the cycles of the restriction of the $i$-th boundary map to the original Koszul complex by $C_i$, then each $C_i$ for $1 \leqslant i \leqslant n$ has a socle summand. 
\end{conjecture}

\let\thefootnote\relax\footnote{
\noindent A\S Subject Classification:\\
Primary: 13C40, 13H10, 14M06, 14M10;
Secondary: 13D02 , 13N05, 14B12, 14M12.\smallbreak
The author is grateful to the
National Science Foundation for partial support.<}
\bibliographystyle{alpha}
\begin{thebibliography}{ABC99}

	\bibitem{DE23} \textbf{Dao, Hailong; Eisenbud, David}. Burch index, summands of syzygies and linearity in resolutions. \textit{Bull. Iranian Math. Soc.} \textbf{49} (2023), no. 2, Paper No. 10, 10 pp. MR4549775

	\bibitem{DKT20} \textbf{Dao, Hailong; Kobayashi, Toshinori; Takahashi, Ryo}. Burch ideals and Burch rings. \textit{Algebra Number Theory} \textbf{14} (2020), no. 8, 2121--2150. MR4172703

	\bibitem{GL69} \textbf{Gulliksen, Tor H.; Levin, Gerson}. Homology of local rings. Queen's Papers in Pure and Applied Mathematics, No. 20 \textit{Queen's University, Kingston, Ont}. 1969 {\rm x}+192 pp. MR0262227

	\bibitem{M2} \textbf{Grayson, Daniel R. and Stillman, Michael E}. Macaulay2, a software system for research in algebraic geometry. Available at \url{http://www.math.uiuc.edu/Macaulay2/}.

	\bibitem{Eis95} \textbf{D. Eisenbud}, Commutative algebra, Graduate Texts in Mathematics, 150, \textit{Springer, New York}, 1995. MR1322960

\end{thebibliography}

\bigskip

\vbox{\noindent Author Addresses:\par
\smallskip
\noindent{David Eisenbud}\par
\noindent{Mathematical Sciences Research Institute,
Berkeley, CA 94720, USA}\par
\noindent{de@msri.org}\par
}

\end{document}


